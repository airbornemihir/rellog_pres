% titlepage-demo.tex
\documentclass{beamer}

% items enclosed in square brackets are optional; explanation below
\title{Relational logic and its applications to verification.}
\subtitle{How to verify several programs at once.}
\author{Mihir Mehta}
\institute{
  Department of Computer Science\\
  University of Texas at Austin\\
  \texttt{mihir@cs.utexas.edu}
}
\date{01 April 2015}

\usepackage{amsmath}
\usepackage{graphicx}

\setbeamertemplate{footline}[frame number]
\begin{document}

%--- the titlepage frame -------------------------%
\begin{frame}
  \titlepage
\end{frame}

%--- the presentation begins here ----------------%
\begin{frame}{Relational logic}
\begin{itemize}
\item Relational logic started out as a means to prove program
  equivalence.
\item Hoare quadruples - intended to express equivalence of two
  programs in a certain context.
\item Benton's initial work - limited to structurally identical
  programs.
\item Later, Zaks and Pnueli, cross products - again limited to
  structurally equivalent programs.
\item Separately, self-composition - sound and complete but gives
  really hard verification conditions.
\item This paper: serves to unite both into one that's actually machine-checkable.
\end{itemize}
\end{frame}

\begin{frame}
\begin{itemize}

\item The program syntax is:
\begin{figure}
\centering
\includegraphics[scale=0.3]{grammar_rule.png}
\end{figure}

\item The program semantics are:
\begin{figure}
\centering
\includegraphics[scale=0.3]{operational_semantics.png}
\end{figure}

\end{itemize}
\end{frame}

\begin{frame}

\begin{figure}
\centering
\includegraphics[scale=0.3]{product_construction_rules.png}
\end{figure}

\begin{figure}
\centering
\includegraphics[scale=0.3]{reduction_rules.png}
\end{figure}

\end{frame}

\begin{frame}[allowframebreaks]
\frametitle{Our work}
\begin{itemize}
\item We'd like to work with proving properties between the input and
  output of several calls to the same function.\\
$ \forall x1, ..., xn. ( Pre(x1, ... , xn) \Rightarrow Post(f(x1),
  ... f(xn))) $
\item By making product programs, we should be able to prove
  properties such as symmetry, reflexivity, transitivity, determinism
  by expressing them as Hoare pre- and post-conditions and then
  passing them to a theorem prover (Z3).
\item Evidently, we can't assume structural equivalence in
  general. For instance, conditionals may go down different branches,
  and loops may have different numbers of iterations.
\item However, we should be able to benefit from the somewhat common
  structure.
\item At present, we're trying to implement a verifier of this kind,
  and also figure out judgement rules that
  might be useful for the kinds of things we're trying to prove.
\item We're focussing on Java programs with comparators - those are
  the main motivating example.
\item We're able to deal with loop-free programs, but will re-write to
  deal better with looping and recursing programs, incorporating the
  product construction.
\end{itemize}
\end{frame}

\end{document}
